\documentclass[11pt,a4j]{jarticle}%jarticleを使用
\usepackage[dvipdfmx]{graphicx}%画像表示の設定
\usepackage{amsmath}%数式周りの強化
\usepackage[all, warning]{onlyamsmath}%eqnarrayを禁止
\usepackage[top=5truemm,bottom=30truemm,left=20truemm,right=20truemm]{geometry}%ページの余白を調整
\usepackage{cite}%引用を整備
\usepackage{ascmac}%screen環境による箱囲みを利用
\usepackage{url}%urlを成形する
\usepackage{listings}%listingによりソースコード表示を補助
%ここまで表示に関する設定

%以下listing時の表示に関する設定
\lstset{
  basicstyle={\ttfamily},
  identifierstyle={\small},
  commentstyle={\smallitshape},
  keywordstyle={\small\bfseries},
  ndkeywordstyle={\small},
  stringstyle={\small\ttfamily},
  frame={tb},
  breaklines=true,
  columns=[l]{fullflexible},
  numbers=left,
  xrightmargin=0zw,
  xleftmargin=3zw,
  numberstyle={\scriptsize},
  stepnumber=1,
  numbersep=1zw,
  lineskip=-0.5ex
}
%ここからタイトル

\begin{document}
\begin{center}
{\huge \bf アセンブラPDF}
\end{center}
\begin{flushright}
ver.1.0 {\bf le3hw} 2020/4/30
\end{flushright}

%ここから本文

\section{アセンブラについて}
本資料は、計算機科学実験及演習3で利用可能なSIMPLE向けアセンブラーに関するものです。命令に関する詳細は、別資料のSIMPLE 設計資料 (ver 4.0: 20200415)を参照してください。
\subsection{使い方}
テキストに命令を実行順で書き下します。引数については以下の表1の順で記載してください。
サンプルテキストとしてsample1.txtを用意してありますのでそちらもご確認ください。
インプットするファイルが準備できたら、
\begin{lstlisting}[label=fuga]
$python3 assembler.py input-file [output-file]
\end{lstlisting}
で、当プログラムを実行します。pythonのバージョンは3.7.6です。output-fileを省略した場合は標準出力に出力されます.
\begin{table}[htbp]
\centering
\caption{input-file内の引数の表記順}
\begin{tabular}{|c|c|c|c|}
\hline
\textbf{命令}  & \textbf{第一引数} & \textbf{第二引数} & \textbf{第三引数} \\ \hline
\textbf{ADD} & R{[}Rd{]}     & R{[}Rs{]}     &               \\ \hline
\textbf{SUB} & R{[}Rd{]}     & R{[}Rs{]}     &               \\ \hline
\textbf{AND} & R{[}Rd{]}     & R{[}Rs{]}     &               \\ \hline
\textbf{OR}  & R{[}Rd{]}     & R{[}Rs{]}     &               \\ \hline
\textbf{XOR} & R{[}Rd{]}     & R{[}Rs{]}     &               \\ \hline
\textbf{CMP} & R{[}Rd{]}     & R{[}Rs{]}     &               \\ \hline
\textbf{MOV} & R{[}Rd{]}     & R{[}Rs{]}     &               \\ \hline
\textbf{SLL} & R{[}Rd{]}     & d             &               \\ \hline
\textbf{SLR} & R{[}Rd{]}     & d             &               \\ \hline
\textbf{SRL} & R{[}Rd{]}     & d             &               \\ \hline
\textbf{SRA} & R{[}Rd{]}     & d             &               \\ \hline
\textbf{IN}  & R{[}Rd{]}     &               &               \\ \hline
\textbf{OUT} & R{[}Rs{]}     &               &               \\ \hline
\textbf{HLT} &            &               &               \\ \hline
\textbf{LD}  & R{[}Ra{]}     & d             & R{[}Rb{]}     \\ \hline
\textbf{ST}  & R{[}Ra{]}     & d             & R{[}Rb{]}     \\ \hline
\textbf{LI}  & R{[}Rb{]}     & d             &               \\ \hline
\textbf{B}   & d             &               &               \\ \hline
\textbf{BE}  & d             &               &               \\ \hline
\textbf{BLT} & d             &               &               \\ \hline
\textbf{BLE} & d             &               &               \\ \hline
\textbf{BNE} & d             &               &               \\ \hline
\end{tabular}
\end{table}

\end{document}